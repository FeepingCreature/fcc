\documentclass[a4paper,10pt,landscape]{article}
\usepackage{multicol}
\usepackage{calc}
\usepackage{ifthen}
\usepackage{verbatim}
\usepackage[landscape]{geometry}
\usepackage[framemethod=tikz]{mdframed}
%\usepackage[style=1]{mdframed}
\usepackage{color}
\usepackage{listings}

\pagestyle{empty}
\definecolor{gray10}{rgb}{0.1,0.1,0.1}
\definecolor{gray90}{rgb}{0.9,0.9,0.9}
\definecolor{code}{rgb}{0.92, 0.92, 0.95}

\newenvironment{segment}[1]
{\begin{mdframed}[roundcorner=5pt]\subsection{#1}}
{\end{mdframed}\vspace{2pt plus 5pt}}
\newcommand{\important}[1]{\textbf{#1}}
\newcommand{\mysection}[1]{
\vspace{0cm plus 5cm}

\section{#1}}

% http://stackoverflow.com/questions/586572/make-code-in-latex-look-nice
\lstset{
language=Java,
basicstyle=\footnotesize,
%numbers=left,
%numberstyle=\footnotesize,
%stepnumber=1,
%numbersep=5pt,
backgroundcolor=\color{code},
showspaces=false,
showtabs=false,
showstringspaces=false,
frame=single,
tabsize=2,
captionpos=b,
boxpos=t,
breaklines=true,
breakatwhitespace=false,
xleftmargin=1em
}

% based on http://www.stdout.org/~winston/latex/
% see there for further doc

\geometry{top=.5in,left=.5in,right=.5in,bottom=.5in}

% Redefine section commands to use less space
\makeatletter
\renewcommand{\section}{\@startsection{section}{1}{0mm}%
			      {-1ex plus -.5ex minus -.2ex}%
			      {0.5ex plus .2ex}%x
			      {\normalfont\large\bfseries}}
\renewcommand{\subsection}{\@startsection{subsection}{2}{0mm}%
			      {-1explus -.5ex minus -.2ex}%
			      {0.5ex plus .2ex}%
			      {\normalfont\normalsize\bfseries}}
\renewcommand{\subsubsection}{\@startsection{subsubsection}{3}{0mm}%
			      {-1ex plus -.5ex minus -.2ex}%
			      {1ex plus .2ex}%
			      {\normalfont\small\bfseries}}
\makeatother

% Don't print section numbers
\setcounter{secnumdepth}{0}

\setlength{\parindent}{0pt}
% \setlength{\parskip}{2pt plus 0.5ex}

% -----------------------------------------------------------------------

\begin{document}

\raggedright
\footnotesize
\begin{multicols}{3}

% multicol parameters
% These lengths are set only within the two main columns
%\setlength{\columnseprule}{0.25pt}
\setlength{\premulticols}{1pt}
\setlength{\postmulticols}{1pt}
\setlength{\multicolsep}{1pt}
\setlength{\columnsep}{2pt}

\begin{center}
    \Large{\textbf{Neat Info}} \\
\end{center}

\begin{segment}{Basic File Structure}
\begin{lstlisting}
// hello.nt
module hello;

void main() {
  writeln "Hello World";
}
\end{lstlisting}

All files begin with \texttt{module}, followed by the name of the module, which is the name
of the file combined with the folder it is in, relative to the root of the folder hierarchy.
For instance, \texttt{src/a/b.nt} is \texttt{module src.a.b} or \texttt{module a.b},
depending on the root.\\

The main file, which is the file that is passed to the compiler,
must contain the \texttt{main} function. The main function returns \texttt{void}
or \texttt{int}, and may take a \texttt{string[]} parameter,
corresponding to the commandline arguments of the program.
\begin{lstlisting}
~ $ fcc hello.nt
\end{lstlisting}
\end{segment}

\begin{segment}{Comments}
The following comments are supported
\begin{itemize}
\item \texttt{//}: rest of the line.
\item \texttt{/* */}: everything in-between is a comment.
\item \texttt{/+ +/}: everything in-between is a comment. Nestable.
\end{itemize}
\end{segment}
\begin{segment}{Index access}
Where \texttt{a} is an array or tuple, a specific element can be accessed using index notation
\texttt{a[i]}, where \texttt{i} is an integer expression. Index access is always zero-based.\\
If \texttt{t} is a tuple, \texttt{a[t]} is equivalent to a tuple of index accesses with the tuple
members. Example: \texttt{a[(2, (3, 4))] == (a[2], (a[3], a[4]))}.\\
If \texttt{r} is a range, \texttt{a[r]} is a \emph{slice}, or sub-array of \texttt{a}.
Slices are start-inclusive and end-exclusive.
Example: \texttt{"Hello World"[6..\$] == "World"}\\
In an index access, \texttt{\$} is the length of the array or tuple.
\end{segment}
\mysection{Types}
Types indicate the memory layout and "meaning" of a value.
\begin{segment}{Basic Types}
The following \important{basic types} exist:\\
\begin{tabular}{@{}ll@{}}
\verb!void!  & The Untype. Indicates the absence of a value.\\
\verb!byte!  & 8-bit signed integer\\
\verb!short! & 16-bit signed integer\\
\verb!int!   & 32-bit signed integer\\
\verb!long!  & 64-bit signed integer\\
\verb!char!  & 8-bit UTF8 code unit\\
\verb!bool!  & 32-bit truth value: either 0 or 1.\\
\verb!float!  & 32-bit floating-point number\\
\verb!double!& 64-bit floating-point number\\
\verb!real!  & \begin{tabular}[x]{@{}l@{}}
Largest native float number. 80-bit on x86.
\end{tabular} \\
\end{tabular}
\end{segment}
\begin{segment}{Derived Types}

There are also \important{derived types}.

\begin{tabular}{@{}ll@{}}
\texttt{T*}  & Pointer to T \\
\texttt{T[]} & Array of T\\
\texttt{T[\textasciitilde]}& Extended Array of T\\
\texttt{T[auto\textasciitilde]}& Managed Array of T\\
\texttt{T function(args)}& Function pointer returning T\\
\texttt{T delegate(args)}& Fat function pointer or delegate.
\end{tabular}

\end{segment}
\begin{segment}{struct}
A \important{struct} is a compound type, declared like so:

\begin{lstlisting}
struct TypeName {
  // these are members
  Type identifier;
  Type2 identifier2, identifier3;
  // this is a member function
  Type func() statement
}
\end{lstlisting}
The size of the struct type is identical to the sum of the size of its members.
Members of a struct can be accessed by name. If \texttt{ex} is an expression of type
\texttt{TypeName}, the member \texttt{identifier} is accessed as \texttt{ex.identifier}.
This also works if \texttt{ex} is a pointer to a struct.

Structs can contain member functions; a member function can access all members of the struct
that contains it.
\end{segment}

\begin{segment}{tuple type}
A \important{tuple type} is something like a free-form struct. It is specified by listing
its component types in parens.

Example tuples:

\begin{lstlisting}
()
(int, int)
(int, float, (string, string))
\end{lstlisting}

There is no such thing as a one-sized tuple.
\end{segment}
\begin{segment}{array}
Arrays are equivalent to
\begin{lstlisting}
struct {
  T* ptr;
  int length;
}
\end{lstlisting}

Extended arrays are equivalent to
\begin{lstlisting}
struct {
  T* ptr;
  int length, capacity;
}
\end{lstlisting}

Capacity indicates the amount of space allocated at the location starting with \texttt{ptr}.
Appending to a managed array makes use of the capacity, where possible; otherwise, new memory
will be allocated.

Managed arrays are equivalent to extended arrays. The difference is, when appending to a managed
array causes a new allocation, the old memory will be freed. This is also known as an "appender"
or "array builder".

\end{segment}

\begin{segment}{\texttt{bool} and \texttt{string}}
\texttt{bool} is a strict alias to \texttt{int}. Neat defines two boolean values,
\texttt{true = bool:1} and \texttt{false = bool:0}.\\
\texttt{string} is just an alias for \texttt{char[]}. Note that the standard library assumes
that all strings are UTF-8. If you are not using UTF-8, you are probably making a mistake.
\end{segment}

\begin{segment}{Conditionals}
Conditionals can be either true or false. \important{Expressions and conditionals are not interchangeable.}
However, since \texttt{!} is used for negation, and \texttt{!expr} is a conditional, the \texttt{!!expr}
pseudo-operator can be used to convert an expression to a conditional, and the \texttt{eval} operator
can be used to explicitly convert a conditional to a boolean expression.

Expressions with a well-defined zero, such as integers, pointers and arrays, are also conditionals.

If \texttt{a} and \texttt{b} are expressions, \texttt{a CMP b} is a conditional, where
\texttt{CMP} may be one of \texttt{== != is < > <= >= <>= !< !> !<= !>= !<>=}, and the meaning of each
operator is equivalent to its comparative symbols "or"ed together. Example: \texttt{!<>=} means "not
smaller, greater or equal", though this is only relevant for exotic float numbers such as NaN.

If \texttt{a} and \texttt{b} are conditionals, \texttt{a CMP b} is a conditional, where \texttt{CMP}
is \texttt{\&\&} or \texttt{||}: boolean "and" and "or". Note that evaluation is short-circuiting.

The difference between \texttt{==} and \texttt{is} is that \texttt{==} checks for \emph{equality},
and  \texttt{is} checks for \emph{identity}. For instance, two strings at separate areas in memory
but with the same content are equal but not identical.

\end{segment}

\clearpage
\mysection{Expressions}
Expressions are things that have a value and a type.
\begin{segment}{Basic Expressions}
Numbers (8) are expressions of type \texttt{int}.

Floating-point numbers (8.5) are expressions of type \texttt{double}.

Floating-point numbers with an \texttt{f} suffix (8.5f) are expressions of type \texttt{float}.

Strings ("Hello World") are expressions of type \texttt{char[]}.

\texttt{true} and \texttt{false} 
\end{segment}

\begin{segment}{implicit conversions}
Numbers, despite being \texttt{int}, implicitly convert to smaller types when they fit.

\begin{lstlisting}
byte b = 125;
\end{lstlisting}

String literals implicitly convert to pointers to char. This is intended to be used with C libraries.
To ensure this is safe, all string literals are zero-terminated. The terminating zero is not
counted as part of the string.

\end{segment}

\begin{segment}{Combined Expressions}

Arithmetic expressions ($a+b$, $a-b$, $a*b$, $a/b$, $a\%b$, $a$ xor $b$, $a \& b$, $a | b$,
$a << b$, $a >> b$, $a >>> b$)
have the type \texttt{int} when the terms ($a$ and $b$) are \texttt{int} or below; otherwise, the type of the
largest term is used.

There are also two unary operators, $!a$ and $-a$, to boolean-negate and numeric-negate respectively.

\end{segment}

\begin{segment}{casts}
Casts are conversions between one type and another.

They take the form: \texttt{Type : expression}

Casts are a "dwim operation": they try to be smart. For instance, casting
float to int or int to float converts the data instead of blindly reinterpreting it.

\texttt{float:2} is a \important{conversion cast}.

Where implicit conversions are possible, casts will use them.

The following are equivalent:

\begin{lstlisting}
byte b = 125;
byte b = byte: 125;
byte b = byte: int: 125;
\end{lstlisting}
The following is \emph{not} equivalent:

\begin{lstlisting}
int i = 125;
byte b = byte: i;
\end{lstlisting}
Though appearing identical, this code actually performs a conversion cast, which may lose data.

As a general rule, \textbf{implicit casts never lose data}.
\end{segment}

\begin{segment}{precedence}

Arithmetic follows the following precedence rules, in order from higher to lower:
\begin{itemize}
\item The "duplicate" operator: $x$
\item The shift operators: $<<$, $>>$ and $>>>$
\item modulo: $\%$
\item binary-and: $\&$
\item binary-or: $|$
\item binary-exclusive or: $xor$
\item multiplication and division: $*$ and $/$
\item addition and subtraction: $+$ and $-$
\end{itemize}
Parens $()$ can be used to force precedence.
\end{segment}
\begin{segment}{\$ and \#}

The $\#$ symbol is equivalent to putting everything left of it into parens:

$a \# b \Rightarrow (a) b$.

The $\$$ symbol is equivalent to putting everything right of it into parens:

$a \$ b \Rightarrow a (b)$.

$\$$ takes precedence over $\#$.
\end{segment}
\begin{segment}{pointer operations}
\texttt{*p} dereferences a pointer. If the type of \texttt{p} is \texttt{T*},
then the type of \texttt{*p} is \texttt{T}.\\
\texttt{\&l} references an expression, creating a pointer. The expression must have an address.
If the type of \texttt{l} is \texttt{T}, the type of \texttt{\&l} is \texttt{T*}.
\end{segment}
\begin{segment}{calls}

Functions, global or nested, struct member functions, class member functions, function pointers
and delegates are all called with this syntax: \texttt{function argument}.

When the function is declared as taking multiple parameters, the argument must be a tuple matching
the types of the function's parameters.

Tuples will be \important{distributed} over the arguments!

For instance, the following code is equivalent:
\begin{lstlisting}
void fun(int a, int b, int c) { }
fun (2, 3, 4);
fun (2, (3, 4));
fun ((2, 3), 4);
\end{lstlisting}

Note that the function can also be written as\\ \texttt{void fun(int a, b, c)}.

\end{segment}

\begin{segment}{variable declarations}
Variable declarations can be used as expressions.
\end{segment}

\begin{segment}{\emph{new} expressions}
\texttt{new} always indicates an allocation of memory.

\begin{itemize}
\item \texttt{new ArrayType size} allocates an array of type \texttt{ArrayType} and size \texttt{size}.
Example: \texttt{new int[] 10}.
\item \texttt{new delegate-literal} allocates the literal as a closure. The stack-bound variables 
it can access are moved to the heap. Note that at present this only operates one level deep.
Example: \texttt{int i = 5; return new \\\{ return i; \};}
\item \texttt{new OtherType} allocates memory for \texttt{OtherType} on the heap, returning a pointer.
\end{itemize}
\end{segment}

\begin{segment}{function pointers and delegates}
The address of a function is taken with the \texttt{\&} operator. This creates a function pointer
with the return type and parameter types of the function.

When taking the address of a class or struct member function, or a nested function, which can access
contextual data (the class or struct, or the surrounding function's stackframe), an additional pointer
must be carried. This creates a \important{delegate} with the return type and parameter types of the 
member function. Delegates are known as "fat pointers" in other languages. Heap-allocated delegates
(see above, \emph{new} expressions) are similar to closures.\\

The following code is equivalent.
\begin{lstlisting}
void main() {
int delegate(int, int) dg1, dg2, dg3;

int add(int a, b) { return a + b; }

dg1 = &add;
dg2 = delegate int(int a, b) {
  return a + b;
} // After a closing '}', a newline
  // can substitute for semicolon.
dg3 = \(int a, b) -> a+b;
}
\end{lstlisting}

\end{segment}

\begin{segment}{tuples}
A tuple value works just like a tuple type. 
\begin{lstlisting}
(int, (int, float)) variable
  = (2, (3, 4.0f));
\end{lstlisting}
The type of a tuple value is the tuple of its values' types. Rolls off the tongue, doesn't it.\\
Note that in order to avoid collision with parens used to disambiguate arithmetic, there is
no such thing as a one-element tuple. The type of \texttt{(4)} is \texttt{int}, not \texttt{(int)}.
\end{segment}

\clearpage
\mysection{Statements}

\begin{segment}{Top-level statements}
Top-level statements are statements that can appear at module level.
\end{segment}

\begin{segment}{Function Declarations}
\important{Function declarations} have the form \texttt{ReturnType functionname([Type parametername[, ...]]) Statement}.

The following two functions are equivalent.

\begin{lstlisting}
int add(int a, int b) { return a + b; }
int add(int a, b) { return a + b; }
\end{lstlisting}

The return type may be \texttt{void}, in which case no value is returned.

Note that \texttt{\{\}} may be omitted if the body of the function consists of a single statement.
For instance, the following two functions are equivalent:
\begin{lstlisting}
void helloworld1()
writeln "Hello World";

void helloworld2() {
writeln "Hello World";
}
\end{lstlisting}
\end{segment}
\begin{segment}{Import}
Modules may \important{import} other modules. The \texttt{import} keyword takes a sequence of
module names, separated by commas. Standard modules have the prefix \texttt{std.} and are located in
the std/ folder. It is an error to import a module twice. When importing a module that is not used,
a warning will be printed. Please avoid importing unnecessary modules.\\
Note that \texttt{import} may also be used inside functions.

\begin{lstlisting}
module hello;
import std.math;
void main() { float f = sin(4); }
\end{lstlisting}
\end{segment}
\begin{segment}{Declarations}
\important{Struct declarations} and class declarations are top-level statements. \\
\important{Variables} declared at module level are \important{global variables}. Their initializers must be constant.

\begin{lstlisting}
module hello;
int globvar;
\end{lstlisting}

\end{segment}

\columnbreak
\begin{segment}{Function-level statements}
These are statements that can appear in functions.\\
The most basic statement is the \emph{aggregate statement}: \texttt{\{\}}.
It may contain any number of statements, which are executed in order.\\
\emph{Expressions} may also be used as statements, although they must be
terminated by a semicolon. Example: \texttt{2+2;} is a statement.\\
\emph{Assignment} takes the form \texttt{a = b;}.
\end{segment}

\begin{segment}{variable declarations}
Variable declarations take the form \texttt{Type identifier}.

Variable declarations may include an \important{initializer} expression, which is assigned
to the variable at the point of declaration. The default initializer is zero.\\
The syntax takes the form \texttt{Type identifier = initializer}.\\
Multiple variables can be declared in a single statement, separated by commas.

The following are equivalent:

\begin{lstlisting}
int a; int b;
int a, b;
int a = 0, b = 0;
auto a = 0, b = 0;
\end{lstlisting}
\texttt{auto} is a \important{placeholder} for the type. When using \texttt{auto}, initializers
\texttt{must} be provided; the type of the variable will be the type of the initializer.\\

There are two alternate keywords to \texttt{auto}, \texttt{ref} and \texttt{scope}.
\texttt{ref} indicates a \emph{reference variable}; it acts like a pointer to the initializer that
is automatically dereferenced for you.
\texttt{scope} is used for variables containing arrays and similar allocated memory. At the end of
the scope surrounding the variable, the memory will be freed.

\begin{lstlisting}
int a;
ref b = a; b = 5;
assert(a == 5);
string s = new string 128;
{ scope var = s; }
// Accessing s now would trigger a segfault.
\end{lstlisting}

\end{segment}
\begin{segment}{using}
\texttt{using} is equivalent to \texttt{with} in Javascript and Pascal. It takes the form
\texttt{using Expression Statement}. \texttt{Expression} is evaluated and saved at the start
of the \texttt{using} block.
When parsing \texttt{Statement}, the properties of \texttt{Expression} are implicitly available.
\end{segment}
\columnbreak
\subsection {control flow statements}
\begin{segment}{while}
The \texttt{while} loop acts as in C. While the conditional evaluates to true, the loop body will be
repeatedly executed. Note that parens for the conditional are not required.
\begin{lstlisting}
  int i;
  while i < 5 { i++; }
\end{lstlisting}
\end{segment}
\begin{segment}{for}
The \texttt{for} loop acts as in C. Note that the comma expression does not exist in Neat!
\begin{lstlisting}
  for (int i = 0; i < 5; ++i) { }
  for (declaration; conditional; statement)
    loopbody;
  // The following code is equivalent
  {
    declaration;
    while (conditional) {
      loopbody;
      statement;
    }
  }
\end{lstlisting}
The \texttt{for} keyword can also be used as a synonym for \texttt{while}.\\

\important{Using a variable declaration with an initializer in a \emph{conditional} that is repeatedly evaluated is undefined behavior.}\\

\end{segment}
\begin{segment}{if}
The \texttt{if} statement acts as in C, Note that parens are still not required.
\begin{lstlisting}
if 2 + 2 == 4
  writeln "sanity still applies";
else
  writeln "abandon all hope";
\end{lstlisting}
\end{segment}
\begin{segment}{in general}

In general, variables declared in a conditional will be valid
in the scope that is executed if the conditional is true.

For instance, a common idiom takes this form:

\begin{lstlisting}
if auto result = someFunction() {
  // result is valid here
} else {
  // but not here
}
\end{lstlisting}

\end{segment}
\pagebreak
\mysection{Neat Stuff}
\textbf{Other Neat Features\footnote{Neatures, if you will.}}

\vspace{1cm}

\begin{segment}{with-tuple}
The syntax \texttt{Expression.(TupleExpression)} is equivalent to \texttt{TupleType result; using (Expression) result = TupleExpression;}\\
Example!
\begin{lstlisting}
struct S {
  int a, b;
}
S s;

s.(a, b) = (2, 3);
// is equivalent to
using s { (a, b) = (2, 3); }

int c, d;

(c, d) = s.(a + b, a - b);
// is equivalent to
using s { (c, d) = (a + b, a - b); }
\end{lstlisting}

Inside a with-tuple, just like inside \texttt{using}, the identifier \texttt{that} references the
stored context expression.

\end{segment}

\begin{segment}{alias}
\texttt{alias} can be used to give expressions and types new names. Expression alias works like
a variable declaration, but the initializer expression is \important{evaluated every time the name is used}.
\begin{lstlisting}
int a = 2, b = 3;
alias c = a + b;
assert(c == 5);
b = 4;
assert(c == 6);
\end{lstlisting}
Note that \texttt{alias} can be used inside struct and class bodies.
\begin{lstlisting}
struct S {
  int a, b;
  alias c = a + b;
}
S s; s.(a, b) = (2, 3);
assert(s.c == 5);
\end{lstlisting}
\end{segment}

\begin{segment}{Iterators}
Iterators are values that can be used as a source of other values.

The simplest iterator is the \emph{Range}: \texttt{From..To}, where both From and To are integer.

Each iterator is equivalent to a conditional that, when evaluated, advances the iterator and evaluates
to the new value.

\begin{lstlisting}
while 0..10
  writeln "Hello";
\end{lstlisting}

The expression used to advance an iterator and return its new value is \texttt{<-}, pronounced
"from". For instance, \texttt{while (int i <- 0..10)}: "while int i from [zero to ten]".
Arrays also behave as iterators.

\begin{lstlisting}
while char ch <- "Hello World"
  writeln "$(int:ch)";
\end{lstlisting}
\end{segment}

\begin{segment}{Scope guards}
Sometimes, a piece of code should be executed when the current scope is exited.

The syntax for this is \texttt{onExit Statement}.

\begin{lstlisting}
int i = 2;
{
  onExit i = 4;
  assert(i == 2);
} // onExit is executed here
assert(i == 4);
\end{lstlisting}

Scope guards are run at three spots: at the closing bracket of the current scope, on \texttt{return},
or when a signal is raised causing an exit to before the scope guard.

To only execute code when the scope is left by a signal, use \texttt{onFailure}. To only execute code
when the scope is left by \texttt{return} or \texttt{\}}, use \texttt{onSuccess}.

\end{segment}

\begin{segment}{Lambdas}
Lambdas, or \emph{anonymous functions}, are nested functions defined via a short-hand syntax
so that no name is required. Lambdas combine function declaration and expression; they are
equivalent to declaring a nested function, then taking its reference with \texttt{\&}.

There are several ways of defining a lambda, in order of verbosity:

\begin{lstlisting}
auto dg1 = delegate ReturnType(Params) Statement;
auto dg2 = \(Params) Statement;
auto dg3 = \(Params) -> ReturnValue;
\end{lstlisting}

For example, these three functions all add an \texttt{int} and a \texttt{float}:

\begin{lstlisting}
// remember, the semicolon for the
// variable add1 can be left out
// because the last line ends on '}'
auto add1 = delegate float(int a, float b)
{ return a + b; }

// Here, the return type is taken from
// the type of the first return statement.
auto add2 = \(int a, float b)
{ return a + b; }

auto add3 = \(int a, float b) -> a + b;
\end{lstlisting}

Furthermore, if the parameter list is empty, it can be left out.

\begin{lstlisting}
auto fun1 = \{ writeln "Hello World"; }
auto fun2 = \-> 5;
\end{lstlisting}

\end{segment}

\end{multicols}
\end{document}
